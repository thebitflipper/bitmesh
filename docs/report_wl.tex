Wireless:

Initial idea and goals The initial goal of this project was to make a
network of cheap radio modules. The inteded use cases were remote
monitoring and home automation. We decided to try to implement a mesh
routing protocol to extend the range of the network.

Initial research and design decisions: 

We started off by reading the
specifications for the intended nrf24l01+ modules. These modules have
126 channels in the 2.4-2.525GHz range, uses gaussian frequency shift
keying and can send 2Mbps, 1Mbps or 250Kbps. In 2Mbps mode the nodes
uses 2Mhz of bandwidth while in 1Mbps and 250Kbps the nodes uses only
1Mhz of bandwidth.  The nodes uses a three, four or five byte long
adress for all packets, can listen to at most six adresses at the same
time where five of the addresses can only differ in the last byte. The
nodes support automatic acknoledgements but with some restrictions,
the adress of the receiving node needs to be in a special address slot
of the transmitting node.  Considering these restrictions we chose to
save the adresses of the neighbours inside an external microcontroller
instead of in the nrf24l01+ chip. This enables us to connect to many
more nodes and makes it easier to enable the automatick
acknowledgement feature


TESTS:
Sleep node: 2mA 
Active: 13mA
